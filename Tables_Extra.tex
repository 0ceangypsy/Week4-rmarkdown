% Options for packages loaded elsewhere
\PassOptionsToPackage{unicode}{hyperref}
\PassOptionsToPackage{hyphens}{url}
%
\documentclass[
]{article}
\usepackage{amsmath,amssymb}
\usepackage{lmodern}
\usepackage{ifxetex,ifluatex}
\ifnum 0\ifxetex 1\fi\ifluatex 1\fi=0 % if pdftex
  \usepackage[T1]{fontenc}
  \usepackage[utf8]{inputenc}
  \usepackage{textcomp} % provide euro and other symbols
\else % if luatex or xetex
  \usepackage{unicode-math}
  \defaultfontfeatures{Scale=MatchLowercase}
  \defaultfontfeatures[\rmfamily]{Ligatures=TeX,Scale=1}
\fi
% Use upquote if available, for straight quotes in verbatim environments
\IfFileExists{upquote.sty}{\usepackage{upquote}}{}
\IfFileExists{microtype.sty}{% use microtype if available
  \usepackage[]{microtype}
  \UseMicrotypeSet[protrusion]{basicmath} % disable protrusion for tt fonts
}{}
\makeatletter
\@ifundefined{KOMAClassName}{% if non-KOMA class
  \IfFileExists{parskip.sty}{%
    \usepackage{parskip}
  }{% else
    \setlength{\parindent}{0pt}
    \setlength{\parskip}{6pt plus 2pt minus 1pt}}
}{% if KOMA class
  \KOMAoptions{parskip=half}}
\makeatother
\usepackage{xcolor}
\IfFileExists{xurl.sty}{\usepackage{xurl}}{} % add URL line breaks if available
\IfFileExists{bookmark.sty}{\usepackage{bookmark}}{\usepackage{hyperref}}
\hypersetup{
  pdftitle={Tables},
  pdfauthor={EE Holmes},
  hidelinks,
  pdfcreator={LaTeX via pandoc}}
\urlstyle{same} % disable monospaced font for URLs
\usepackage[margin=1in]{geometry}
\usepackage{color}
\usepackage{fancyvrb}
\newcommand{\VerbBar}{|}
\newcommand{\VERB}{\Verb[commandchars=\\\{\}]}
\DefineVerbatimEnvironment{Highlighting}{Verbatim}{commandchars=\\\{\}}
% Add ',fontsize=\small' for more characters per line
\usepackage{framed}
\definecolor{shadecolor}{RGB}{248,248,248}
\newenvironment{Shaded}{\begin{snugshade}}{\end{snugshade}}
\newcommand{\AlertTok}[1]{\textcolor[rgb]{0.94,0.16,0.16}{#1}}
\newcommand{\AnnotationTok}[1]{\textcolor[rgb]{0.56,0.35,0.01}{\textbf{\textit{#1}}}}
\newcommand{\AttributeTok}[1]{\textcolor[rgb]{0.77,0.63,0.00}{#1}}
\newcommand{\BaseNTok}[1]{\textcolor[rgb]{0.00,0.00,0.81}{#1}}
\newcommand{\BuiltInTok}[1]{#1}
\newcommand{\CharTok}[1]{\textcolor[rgb]{0.31,0.60,0.02}{#1}}
\newcommand{\CommentTok}[1]{\textcolor[rgb]{0.56,0.35,0.01}{\textit{#1}}}
\newcommand{\CommentVarTok}[1]{\textcolor[rgb]{0.56,0.35,0.01}{\textbf{\textit{#1}}}}
\newcommand{\ConstantTok}[1]{\textcolor[rgb]{0.00,0.00,0.00}{#1}}
\newcommand{\ControlFlowTok}[1]{\textcolor[rgb]{0.13,0.29,0.53}{\textbf{#1}}}
\newcommand{\DataTypeTok}[1]{\textcolor[rgb]{0.13,0.29,0.53}{#1}}
\newcommand{\DecValTok}[1]{\textcolor[rgb]{0.00,0.00,0.81}{#1}}
\newcommand{\DocumentationTok}[1]{\textcolor[rgb]{0.56,0.35,0.01}{\textbf{\textit{#1}}}}
\newcommand{\ErrorTok}[1]{\textcolor[rgb]{0.64,0.00,0.00}{\textbf{#1}}}
\newcommand{\ExtensionTok}[1]{#1}
\newcommand{\FloatTok}[1]{\textcolor[rgb]{0.00,0.00,0.81}{#1}}
\newcommand{\FunctionTok}[1]{\textcolor[rgb]{0.00,0.00,0.00}{#1}}
\newcommand{\ImportTok}[1]{#1}
\newcommand{\InformationTok}[1]{\textcolor[rgb]{0.56,0.35,0.01}{\textbf{\textit{#1}}}}
\newcommand{\KeywordTok}[1]{\textcolor[rgb]{0.13,0.29,0.53}{\textbf{#1}}}
\newcommand{\NormalTok}[1]{#1}
\newcommand{\OperatorTok}[1]{\textcolor[rgb]{0.81,0.36,0.00}{\textbf{#1}}}
\newcommand{\OtherTok}[1]{\textcolor[rgb]{0.56,0.35,0.01}{#1}}
\newcommand{\PreprocessorTok}[1]{\textcolor[rgb]{0.56,0.35,0.01}{\textit{#1}}}
\newcommand{\RegionMarkerTok}[1]{#1}
\newcommand{\SpecialCharTok}[1]{\textcolor[rgb]{0.00,0.00,0.00}{#1}}
\newcommand{\SpecialStringTok}[1]{\textcolor[rgb]{0.31,0.60,0.02}{#1}}
\newcommand{\StringTok}[1]{\textcolor[rgb]{0.31,0.60,0.02}{#1}}
\newcommand{\VariableTok}[1]{\textcolor[rgb]{0.00,0.00,0.00}{#1}}
\newcommand{\VerbatimStringTok}[1]{\textcolor[rgb]{0.31,0.60,0.02}{#1}}
\newcommand{\WarningTok}[1]{\textcolor[rgb]{0.56,0.35,0.01}{\textbf{\textit{#1}}}}
\usepackage{graphicx}
\makeatletter
\def\maxwidth{\ifdim\Gin@nat@width>\linewidth\linewidth\else\Gin@nat@width\fi}
\def\maxheight{\ifdim\Gin@nat@height>\textheight\textheight\else\Gin@nat@height\fi}
\makeatother
% Scale images if necessary, so that they will not overflow the page
% margins by default, and it is still possible to overwrite the defaults
% using explicit options in \includegraphics[width, height, ...]{}
\setkeys{Gin}{width=\maxwidth,height=\maxheight,keepaspectratio}
% Set default figure placement to htbp
\makeatletter
\def\fps@figure{htbp}
\makeatother
\setlength{\emergencystretch}{3em} % prevent overfull lines
\providecommand{\tightlist}{%
  \setlength{\itemsep}{0pt}\setlength{\parskip}{0pt}}
\setcounter{secnumdepth}{-\maxdimen} % remove section numbering
\usepackage{booktabs}
\usepackage{longtable}
\usepackage{array}
\usepackage{multirow}
\usepackage{wrapfig}
\usepackage{float}
\usepackage{colortbl}
\usepackage{pdflscape}
\usepackage{tabu}
\usepackage{threeparttable}
\usepackage{threeparttablex}
\usepackage[normalem]{ulem}
\usepackage{makecell}
\usepackage{xcolor}
\ifluatex
  \usepackage{selnolig}  % disable illegal ligatures
\fi

\title{Tables}
\author{EE Holmes}
\date{}

\begin{document}
\maketitle

This is a more complex example of making tables.

\begin{Shaded}
\begin{Highlighting}[]
\FunctionTok{library}\NormalTok{(knitr)}
\FunctionTok{library}\NormalTok{(kableExtra)}
\FunctionTok{library}\NormalTok{(xtable)}
\NormalTok{output.type }\OtherTok{\textless{}{-}}\NormalTok{ knitr}\SpecialCharTok{::}\NormalTok{opts\_knit}\SpecialCharTok{$}\FunctionTok{get}\NormalTok{(}\StringTok{\textquotesingle{}rmarkdown.pandoc.to\textquotesingle{}}\NormalTok{)}
\NormalTok{output.type}
\end{Highlighting}
\end{Shaded}

\begin{verbatim}
## [1] "latex"
\end{verbatim}

Here I'll use a data frame from from a R data set, but in practice your
data frame would come from a function.

Note the first time you make a PDF, tinytex will load any needed
packages and it can take a \textbf{long} time. You'll see a spinning
wheel on the R Markdown tab.

\begin{Shaded}
\begin{Highlighting}[]
\NormalTok{a }\OtherTok{\textless{}{-}}\NormalTok{ mtcars[}\DecValTok{1}\SpecialCharTok{:}\DecValTok{7}\NormalTok{, }\DecValTok{1}\SpecialCharTok{:}\DecValTok{6}\NormalTok{]}
\end{Highlighting}
\end{Shaded}

Default \textbf{kable} table.

\begin{Shaded}
\begin{Highlighting}[]
\NormalTok{knitr}\SpecialCharTok{::}\FunctionTok{kable}\NormalTok{(a)}
\end{Highlighting}
\end{Shaded}

\begin{tabular}{l|r|r|r|r|r|r}
\hline
  & mpg & cyl & disp & hp & drat & wt\\
\hline
Mazda RX4 & 21.0 & 6 & 160 & 110 & 3.90 & 2.620\\
\hline
Mazda RX4 Wag & 21.0 & 6 & 160 & 110 & 3.90 & 2.875\\
\hline
Datsun 710 & 22.8 & 4 & 108 & 93 & 3.85 & 2.320\\
\hline
Hornet 4 Drive & 21.4 & 6 & 258 & 110 & 3.08 & 3.215\\
\hline
Hornet Sportabout & 18.7 & 8 & 360 & 175 & 3.15 & 3.440\\
\hline
Valiant & 18.1 & 6 & 225 & 105 & 2.76 & 3.460\\
\hline
Duster 360 & 14.3 & 8 & 360 & 245 & 3.21 & 3.570\\
\hline
\end{tabular}

This is the table with \textbf{xtable} for LaTeX and \textbf{kableExtra}
for html.

\begin{table}[h]
\centering
\caption{My table caption 2021-06-29} 
\begin{tabular}{r|c|cccc}
  \hline
mpg & cyl & disp & hp & drat & wt \\ 
  \hline
21 & 6 & 160 & 110 & 4 & 3 \\ 
  21 & 6 & 160 & 110 & 4 & 3 \\ 
  23 & 4 & 108 & 93 & 4 & 2 \\ 
  21 & 6 & 258 & 110 & 3 & 3 \\ 
  19 & 8 & 360 & 175 & 3 & 3 \\ 
  18 & 6 & 225 & 105 & 3 & 3 \\ 
  14 & 8 & 360 & 245 & 3 & 4 \\ 
   \hline
\end{tabular}
\end{table}
\begin{landscape}\begin{table}[p]
\centering
\caption{My table caption 2021-06-29} 
\begin{tabular}{r|c|cccc}
  \hline
mpg & cyl & disp & hp & drat & wt \\ 
  \hline
21 & 6 & 160 & 110 & 4 & 3 \\ 
  21 & 6 & 160 & 110 & 4 & 3 \\ 
  23 & 4 & 108 & 93 & 4 & 2 \\ 
  21 & 6 & 258 & 110 & 3 & 3 \\ 
  19 & 8 & 360 & 175 & 3 & 3 \\ 
  18 & 6 & 225 & 105 & 3 & 3 \\ 
  14 & 8 & 360 & 245 & 3 & 4 \\ 
   \hline
\end{tabular}
\end{table}
\end{landscape}

\end{document}
